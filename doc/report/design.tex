\section{System Design}
\label{sec:design}

Our system consists of a mobile client with two Wi-Fi radio interfaces. 
As Figure~\ref{fig:topo} shows, when the mobile client moves away from AP-1 towards AP-2, 
the quality of the wireless link between the client and AP-1 degrades and 
that of the link between the client and AP-2 improves.

In order to accomplish lossless hand-over, we need a way to reroute the ongoing flows
as the client moves from one AP to another. \sys{} achieves this using the
NOX controller.

\begin{figure*}[htb!]
\begin{minipage}[t]{0.45\textwidth}
\centerline{\includegraphics[width=\columnwidth]{fig/topo}}
\caption{OpenFlow network with mobile client.}
\label{fig:topo}
\end{minipage}
\hfill
\begin{minipage}[t]{0.45\textwidth}
\centerline{\includegraphics[width=\columnwidth]{fig/system_arch}}
\caption{\sys{} system design.}
\label{fig:sys}
\end{minipage}
\hfill
\end{figure*}

Traditional mobile systems with a single radio interface suffer significant 
degradation in performance during hard handoff from one AP to another. 
However, the configuration with two radio interfaces provides the 
possibility of seamless hand-over between APs.

\sys{} operates based on a 4-stage protocol, which is described below:

\begin{itemize}

\item
{\bf MAKE\_REQ} : The client is using one radio (\emph{radio-1}) for transmission, while using 
the other radio (\emph{radio-2}) to monitor the quality of the available wireless links (eg. Signal to Noise Ratio) 
to the surrounding APs. When the link quality degrades below a certain threshold, 
the client issues a MAKE\_REQ to the NOX controller.

\item
{\bf MAKE\_ACK} : The NOX controller prepares itself for a \emph{flow-in} event from the client,
and acknowledges the same to the client using MAKE\_ACK.

\item
{\bf BREAK\_REQ} : The client connects to the new AP using \emph{radio-2}, and 
sends all its outgoing traffic over it. However, it continues to receive packets from the old AP
over \emph{radio-1}. As the first outgoing packet reaches the new AP, it triggers a flow-in event
at the NOX controller. The NOX controller, now, looks up its flow-database, computes the common root-switches
 and reroutes all the 
flows destined to the client to go over the new AP. Simultaneously, the client issues a BREAK\_REQ to NOX.

\item
{\bf BREAK\_ACK} : After the controller gets a BREAK\_REQ message from the client, it waits for a predefined
\emph{flush-time} (which equals RTT/$2$), so that all the packets remaining on the old route reach the client.
It then sends the BREAK\_ACK message to the client, upon the receipt of which, the client dissociates \emph{radio-1} from
the old AP and uses it for background scanning.

\end{itemize}

